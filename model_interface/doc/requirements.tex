\documentclass{article}
\begin{document}
\maketitle {An Eiffel Model Interface (EMI)}
\section {Design by Contract}
Design by contracts (DBC) is a significant advance in software engineering. Once you practice it you can't imagine working without it anymore. The concept is that a contract is made between a provider of services and a consumer of those services. The consumer has to satisfy so-called preconditions before asking a service. Once the preconditions are met the provider is contractually tied to ensure the postconditions of the service.  Invariants describe conditions that apply everytime, i.e. before and after any service call.

Providers and consumer have rights and obligations. Benefits are also for both. The consumer can rely without ambiguity on the offered service, and the provider knows under wich conditions it must offer the service. This helps for the specifications, design, coding and testing phases. DBC is therefore a very good practice.
\section {Redundancy}
Another good practice is to avoid redundancy. Redundancy can appear in every area : specification, design, code-fragments, etc ... Redundancy has some severe drawbacks. Besides the fact that you have to spend time duplicating the same part several times, redundant parts get out of sync quicker than you can expect. One of the nightmares of software engineerings is not knowing anymore which parts should be the same, and even worse, what is finally the right version among all those redundant parts.
\section {Redundancy and Design by Contract}
Provider redundancy is avoided by using DBC and object oriented sofware construction : the service is defined once by its contract in a class. 

Consumer redundancy depends on the kind of consumer. We shall make here a distinction between consumers that are internal or external relatif to a system.

There are two types of ``consumership'' : external and internal.
An external consumer does not know software contracts, for example a text widget that ``should'' contain a date.  An internal consumer does know about contratcs; this is the code that should create a valid instance of the DATE datatype.

\section {Internal consumers}
Internal consumers follows the DBC principle strictly,  they have been build with the contracts in mind. Examples of internal consumers are instances of classes that are part of the model of the system.
Redundancy is low for internal consumers. The consumer will only invoke services when the pre-conditions are  met.
\section {External consumers}
External consumers are not build with the system contracts in mind. Either they don't know the contracts or can't fullfill the pre-conditions. They typically invoke services without verifying the preconditions. To avoid a system-crash some kind of *watchdog*, that cares about contracts, must be put between external consumers and internal providers. Examples of external consumers are ``humans'', software systems, input devices, databases, ...

Redundancy is rather high when the service is invoked by an external consumer. Each watchdog must check the whole precondition before invoking the service. When the preconditions aren't met the interface shouldn't invoke the service and inform the consumer in some way. This information should describe why the service couldn't be invoked. In the life-time of the system you must repeat  all thos checks/reports for all possible consumers and for each service. The number of different consumers of a service can become important over time :  variations on data-sources, platforms, GUI's, technologies, hardware, software systems

\section {Requirement for an eiffel model interface}

The basic requirement here is to provide a way to minimize redundancy in the external consumer-interfaces. A system build with this feature in mind could be extended without pain to other external consumers without risks to break the system contracts.
\section {Requirement validation}
A typical application of the requirement offers a single consumer-interface for a service used in a charactermode, GUI and web environment.  
    
\end{document}
